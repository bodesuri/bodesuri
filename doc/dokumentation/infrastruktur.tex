\documentclass[a4paper,12pt,halfparskip,DIV14]{scrreprt}

\usepackage{ngerman}
\usepackage[utf8]{inputenc}
\usepackage{lmodern}
\usepackage[T1]{fontenc}
\usepackage{textcomp}
\usepackage[pdftex, colorlinks, linkcolor=blue]{hyperref}
\usepackage[pdftex]{graphicx}
\usepackage{booktabs}
\usepackage{xcolor}

\usepackage[automark,headsepline]{scrpage2}

\pagestyle{scrheadings}
\renewcommand*{\chapterpagestyle}{scrheadings}

\clearscrheadfoot
\chead{\normalfont\sffamily \headmark}
\cfoot{\pagemark}

\usepackage{enumitem}
\setlist{itemsep=-2pt}


\begin{document}
% FIXME: header passt mir noch nicht
\title{Bodesuri\\Infrastruktur}
\subject{Infastruktur-Dokumentation}
\author{Philipe~Eberli, Pascal~Hobus, \\ Danilo~Couto, Reto~Schüttel, Robin~Stocker}
\maketitle

\pagenumbering{roman}

\tableofcontents
\thispagestyle{plain}
\newpage

\pagenumbering{arabic}

\chapter{Build-Server} % (fold)
\label{cha:build_server}

Bei dem Build-Server handelt es sich um ein virtueller Server welcher freundlicherweise für die Dauer des Projektes gesponsert durch Speed PC gesponsert wurde. Ein virtuelles System bietet dem Projekt beinahe alle Vorzüge eines richtigen dedizierten physikalischen Systems, erspart uns aber die sonst üblichen Aufwendungen wie Platzierung, Wartung etc. die durch den Dienstleister (in diesem Fall Speed PC) übernommen werden. Im Unterschied zu einer Shared-Hosting Lösung (Subversion Server HSR oder Variante Philipp Mächler) muss bei ein dem virtuellen Server keine Rücksicht auf andere Parteien genommen werden und das System lässt sich ganz den Bedürfnissen anpassen.


\section{Wichtige Passwörter} % (fold)
\label{sec:wichtige_passwörter}

\begin{description}
  \item[mysql] User: root Passwort: bodesuri
%FIXME: cruisecontrol
\end{description}

% section wichtige_passwörter (end)

\section{Netzwerk} % (fold)
\label{sec:netzwerk}

Das System ist über den Hostnamen \textbf{bodesuri.speedpc.ch} erreichbar. 

\begin{description}
  \item[Hostname] bodesuri.speedpc.ch
  \item[IP Adresse] 195.49.105.242
\end{description}

Das Hostsystem steht in einem Rechenzentrum der Via.Networks in Zürich.

% section netzwerk (end)

\section{Soft- und Hardware} % (fold)
\label{sec:soft_und_hardware}

Der virtuelle Server besitzt 512 MB Memory und es steht ihm 2 GB Diskspace zur Verüfgung. Das Hostsystem läuft auf einem 2x Intel Xeon 3.2 GHZ System mit 2 GB Memory. Der virtuelle Server steht dediziert für das Projekt zur Verfügung.

Die Randdaten des virtuellen Servers (Disk \& Memory) können bei Bedarf sofort erhöht werden, solange die Ressourcen des Hostsystems nicht erschöpft sind geht das ohne Probleme und ohne physikalischen Eingriff. Genügt hingegen die Prozessorleistung nicht muss eine Umstieg auf einen physikalisch dediziertes System in Erwägung gezogen werden.

Auf dem Virtuellen Server läuft ein frisch installiertes Debian 4.0 (Etch) mit allen Sicherheitsupdates installiert.

% section soft_und_hardware (end)

\section{System} % (fold)
\label{sec:system}

% section system (end)

\section{Benutzeraccounts} % (fold)
\label{sec:benutzeraccounts}

Es wurden für alle Teammitglieder ein Benutzeraccount erstellt. Das Passwort wurde per E-Mail mitgeteilt und muss beim ersten Login geändert werden.

\begin{itemize}
  \item phobus
  \item peberli
  \item ecouto
  \item rstocker
  \item rschuett
\end{itemize}

Die Benutzeraccounts wurden folgendermassen erstellt. 

\begin{verbatim}
# adduser <username>
# passwd -e <username>
\end{verbatim}

% section benutzeraccounts (end)

\section{Verwaltung} % (fold)
\label{sec:verwaltung}

Alle Teammitglieder sind der Gruppe wheel zugeordnet, welche ihnen erlaubt per \textbf{sudo} administrative Arbeiten durchzuführen. Es wird davon abgeraten direkt mit dem Benutzer root zu arbeiten.

\begin{verbatim}
# cat /etc/sudoers 

# use visudo to edit this file!

root    ALL=(ALL) ALL
%wheel  ALL=(ALL) ALL

# addgroup --system wheel
Adding group `wheel' (GID 101) ...
Done.

# adduser <username> wheel
\end{verbatim}


% section verwaltung (end)

% chapter build_server (end)

\chapter{Tools} % (fold)
\label{cha:tools}

\section{Subversion} % (fold)
\label{sec:subversion}

\subsection{Installation} % (fold)
\label{sub:installation-subversion}

Benötigte Software installieren.

\begin{verbatim}
# aptitude install apache2 libapache2-svn subversion subversion-tools
\end{verbatim}

Für CruiseControl muss \textbf{ant} auf Subverison-Repositories zugreiffen können. Hierfür wird die Erweiterung SvnAnt benötigt. http://subclipse.tigris.org/svnant.html

\begin{verbatim}
# aptitude install libsvn-java
# cd /tmp
# wget http://subclipse.tigris.org/files/documents/906/32898/svnant-1.0.0.zip
# unzip svnant-1.0.0.zip
# cd svnant-1.0.0
# ant
# cd ..
# mv svnant-1.0.0 /opt/
# ln -s svnant svnant-1.0.0
\end{verbatim}

Initiales Repository einrichten .

\begin{verbatim}
cd /home
mkdir svn
chown www-data svn
sudo -u www-data svnadmin create svn
\end{verbatim}

Konfiguration Apache in \textbf{/etc/apache2/conf.d/subversion}

\begin{verbatim}
<Location /svn>
        DAV svn
        SVNPath /home/svn
        AuthType Basic
        AuthName "Bodesuri Subversion Repository"
        AuthUserFile /etc/apache2/dav_svn.passwd
        Require valid-user
</Location>
\end{verbatim}

Einrichten der Subversion User

\begin{verbatim}
# htpasswd -bcm /etc/apache2/dav_svn.passwd rschuett bodesuri321
# htpasswd -bm  /etc/apache2/dav_svn.passwd rstocker bodesuri321
# htpasswd -bm  /etc/apache2/dav_svn.passwd ecouto   bodesuri321
# htpasswd -bm  /etc/apache2/dav_svn.passwd phobus   bodesuri321
# htpasswd -bm  /etc/apache2/dav_svn.passwd peberli  bodesuri321
\end{verbatim}

Apache neu starten

\begin{verbatim}
# /etc/init.d/apache2 restart
\end{verbatim}

Zum testen kann das Repository im Browser geöffnet werden: \textbf{http://bodesuri.speedpc.ch/svn/} 

% subsection installation (end)

% section subversion (end)

\section{Java SDK 1.5} % (fold)
\label{sec:java_sdk_1_5}

\begin{verbatim}
# aptitude install sun-java5-jdk
\end{verbatim}

% section java_sdk_1_5 (end)

\section{Java SDK 1.6} % (fold)
\label{sec:java_sdk}

Für Debian Etch ist kein Java 1.6 erhältlich (nur 1.5) - aus diesem Grund muss Java semi-manuell installiert werden. Anstatt Java direkt \emph{von hand} zu installieren kann das Tool \textbf{java-package} zu Hilfe genommen werden. Es wandelt die Java Pakete von Sun in richtige Debian Pakete um. 

Installation java-package

\begin{verbatim}
# aptitude install java-package  
\end{verbatim}

\begin{verbatim}
$ cd /tmp
$ mkdir java-package
$ cd java-package
$ apt-get source java-package
$ cd java-package-0.28/
$
\end{verbatim}

Anschliessend muss das neuste Java 6 SDK (Version: \textbf{Linux self-extracting file}) von \textbf{http://java.sun.com/javase/downloads/index.jsp} heruntergeladen werden.


Mittels dem Befehl \textbf{make-jpkg} wird das Paket nun umgewandelt. Es ist wichtig, dass dieser Schritt mit einem normalen Benutzer durchgeführt wird (nicht mit root).

\begin{verbatim}
$ fakeroot make-jpkg jdk-6u1-linux-i586.bin 
\end{verbatim}

% section java_sdk (end)

\section{Tomcat 5} % (fold)
\label{sec:tomcat_5}

Für das Reporting durch CruiseControl wird Apache2/Tomcat benötigt. Installation über die Pakete von Debian

\begin{verbatim}
# aptitude install tomcat5 libapache2-mod-jk
\end{verbatim}

% section tomcat_5 (end)

\section{CruiseControl} % (fold)
\label{sec:cruisecontrol}

\subsection{Installation} % (fold)
\label{sub:installation-cruisecontrol}


Dependencies installieren.

\begin{verbatim}
# aptitude install ant junit junit-doc
\end{verbatim}

Download von CruiseControl.

\begin{verbatim}
# cd /tmp
# wget 'http://heanet.dl.sourceforge.net/sourceforge/cruisecontrol/cruisecontrol-src-2.6.1.zip'
\end{verbatim}

Kompilierung.

\begin{verbatim}
# cd cruisecontrol-2.6.1/
# cd main
# ant
# cd ../reporting/jsp/
# ant
\end{verbatim}

Jeweils nach den Aufrufen von \textbf{ant} sollte anhand des Outputes geprüft werden ob der Build erfolgreich war.

Abschliessend muss cruisecontrol nach an seine finale Position verschoben werden.

\begin{verbatim}
# mv /tmp/cruisecontrol-2.6.1/ /opt/
# cd /opt
# ln -s cruisecontrol cruisecontrol-2.6.1
\end{verbatim}


% subsection installation (end)

\subsection{Konfiguration Builddaemon} % (fold)
\label{sub:konfiguration_builddaemon}

Erstellen des Cruisecontrol Users und dessen Workdir

\begin{verbatim}
# adduser --system --disabled-password --disabled-login cruisecontrol
# cd /home/cruisecontrol/
# sudo -u cruisecontrol mkdir -p logs/bodesuri checkout/bodesuri artifacts
\end{verbatim}

Checkout des Repositories für den Builddaemon. 

\begin{verbatim}
# htpasswd -bm  /etc/apache2/dav_svn.passwd cruisecontrol bodesuri321
# sudo -u cruisecontrol svn checkout --username cruisecontrol --password bodesuri321 http://bodesuri.speedpc.ch/svn bodesuri/
\end{verbatim}


% subsection konfiguration_builddaemon (end)

\subsection{Konfiguration Reporting} % (fold)
\label{sub:konfiguration_reporting}

Stoppen der Tomcat Engine.

\begin{verbatim}
# /etc/init.d/tomcat5 stop
\end{verbatim}

Reporting in den Tomcat Applikationsfolder kopieren.

\begin{verbatim}
# cp /opt/cruisecontrol/reporting/jsp/dist/cruisecontrol.war /var/lib/tomcat5/webapps/
\end{verbatim}

Tomcat wieder starten, es sollte nun das .war File entapcken.

\begin{verbatim}
# /etc/init.d/tomcat5 stat
\end{verbatim}

Anschliessend müssen folgende Felder in der Datei \textbf{/var/lib/tomcat5/webapps/cruisecontrol/WEB-INF/web.xml} angepasst werden:

\begin{description}
  \item[logDir] /home/cruisecontrol/logs/
  \item[rootDir] /home/cruisecontrol/artifacts/
  \item[currentBuildStatusFile] buildstatus.txt
\end{description}

Zusätzlich benötigt das Reporting Modul Zugriffsrechte auf den Ordner \textbf{\_cache} im Log-Ordner.

\begin{verbatim}
# mkdir logs/bodesuri/_cache/
# chown tomcat5 logs/bodesuri/_cache/
\end{verbatim}

% subsection konfiguration_reporting (end)

% section cruisecontrol (end)

\section{MySQL} % (fold)
\label{sec:mysql}

Für Trac und Dot-Project wird MySQL Benötigt. Installation von MySQL über Debian-Pakete.

\begin{verbatim}
# aptitude install mysql-server
\end{verbatim}

Einrichtung von MySQL.

\begin{verbatim}
# mysql -u root
> use mysql;
> update user set password = password("bodesuri") where user = "root";
> flush privileges;
\end{verbatim}

% section mysql (end)

\section{Dot-Project} % (fold)
\label{sec:dot_project}

Dot-Project kann von \url{http://www.dotproject.net/} heruntergeladen werden.

Einrichten 

% section dot_project (end)

% chapter tools (end)

\chapter{Backup} % (fold)
\label{cha:backup}

% chapter backup (end)

\end{document}
